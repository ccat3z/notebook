\documentclass[main.tex]{subfiles}
\begin{document}
\chapter{基础概念}
\section{基础}
\begin{enumerate}
    \item 随机实验 $E: \{\text{样本点}, \ldots\}$
    \item 样本空间 $S \subseteq E$
        \begin{itemize}
            \item 必然事件
            \item 随机事件
            \item 不可能事件 $\emptyset$
        \end{itemize}
    \item 事件间关系及运算
        \begin{itemize}
            \item 包含 $\subset$ $\subseteq$
            \item 等于 $=$
            \item 和事件 $A\cup B$,
                         $\bigcup_{k=1}^n A_k$,
                         $\bigcup_{k=1}^\infty A_k$
            \item 积事件 $A\cap B$,
                         $\bigcap_{k=1}^n A_k$,
                         $\bigcap_{k=1}^\infty A_k$
            \item 差事件 $A - B$
            \item 互不相容 (互斥) $A \cap B = \emptyset$
            \item 逆事件 (对立事件) $A \cup B = S \land A \cap B = \emptyset$
            \item 交换律
            \item 结合律
            \item 分配律
            \item 德摩根律 $\overline{A \cup B} = \bar{A} \cap \bar{B}$
        \end{itemize}
    \item 频率
        \begin{itemize}
            \item 频数 $n_A$: n次试验中A发生到次数
            \item 频率 $f_n(A) = n_A / n$
                \begin{itemize}
                    \item $0 \leq f_n(A) \leq 1$
                    \item $f_n(S) = 1$
                    \item $A_1, A_2, \ldots, A_k \text{两两互不相容}
                           \Rightarrow
                           f_n(A_1 \cup A_2 \cup \ldots \cups A_k) =
                           f_n(A_1) + f_n(A_2) + \ldots f_n(A_k)$
                    \item $n \rightarrow \infty
                           \Rightarrow
                           f_n(A) \rightarrow P(A)$
                \end{itemize}
        \end{itemize}
    \item 概率 $P(A)$
        \begin{itemize}
            \item 非负数 $P(A) \geq 0$
            \item 规范性 $P(S) = 1$ 性质扩展 $P(A) \leq P(S) = 1$
            \item 可列可加性质, 有限可加性质
                  $A_1, A_2, \ldots \text{两两互不相容}
                   \Rightarrow
                   f_n(A_1 \cup A_2 \cup \ldots) =
                   f_n(A_1) + f_n(A_2) + \ldots$
            \item $P(\emptyset) = 0$
            \item $A \subset B \Rightarrow P(B - A) = P(B) - P(A), P(B) > P(A)$
            \item $P(\bar{A}) = 1 - P(A)$
            \item \underline{加法公式 $P(A \cup B) = P(A) + P(B) - P(AB)$}
        \end{itemize}
\end{enumerate}

\section{等可能概型 (古典概型)}
\begin{enumerate}
    \item 样本空间包含有限个元素
    \item 每个事件到可能性相同
\end{enumerate}
\begin{define}[组合数]\label{组合数}
    \[r \leq a, C_r^a = \vectornum{a}{r} = \dfrac{a (a - 1) \ldots (a - r + 1)}{r!}\]
\end{define}

\section{条件概率}
\begin{enumerate}
    \item 条件概率
        \begin{mthm}[条件概率]\label{条件概率}
            A事件发生当条件下B发生当条件概率
            \[P(B | A) = \dfrac{P(AB)}{P(A)}\]
        \end{mthm}
    \item \uwave{乘法定理}
        \begin{mthm}[乘法定理]\label{乘法定理}\label{乘法公式}
            \[P(AB) = P(B \mid A)P(A)\]
        \end{mthm}
        由乘法公式~\ref{乘法公式}推导可得:
        \[P(ABC) = P(C \mid AB)P(B \mid A)P(A)\]
    \item \uwave{全概率公式}
        \begin{define}[划分]\label{划分}
            $S$为$E$的样本空间, $B_1, B_2, \ldots, B_n$ 为$E$的一组事件,
            满足 $B_i B_j = \emptyset, i \not= j, i, j = 1, 2, \ldots, n \land B_1 \cup B_2 \cup \ldots \cup B_n = S$,
            则称 $B_1, B_2, \ldots, B_n$ 为$E$的一个划分
        \end{define}
        \begin{mthm}[全概率公式]\label{全概率公式}
            $B_1, B_2, \ldots, B_n$ 为S当一个划分~\ref{划分}
            \[P(A) = P(A \mid B_1)P(B_1) + P(A \mid B_2)P(B_2) + \cdots + P(A \mid B_n)P(B_n)\]
        \end{mthm}
    \item \uwave{贝叶斯公式}
        \begin{mthm}[贝叶斯公式]\label{贝叶斯公式}
            $B_1, B_2, \ldots, B_n$为$S$的一个划分~\ref{划分},
            由条件概率~\ref{条件概率}及全概率公式~\ref{全概率公式}推得
            \[
                P(B_i \mid A) = \frac{P(B_i A)}{P(A)} = \frac{P(A \mid B_i)P(B_i)}{\sum_{j=1}^{n}P(A \mid B_j)P(B_j)}
            \]
        \end{mthm}
\end{enumerate}

\section{独立性}
\begin{define}[独立]\label{独立}\label{相互独立}
    对于$A, B$两事件
    \[ P(AB) = P(A)P(B) \Leftrightarrow A, B \text{相互独立}, \text{简称} A, B \text{独立} \]
\end{define}
\begin{mthm}
    \[ A, B \text{独立} \Leftrightarrow A, \bar{B} \text{独立} \]
\end{mthm}
\begin{define}[多事件独立]
    \[
        \left.
            \begin{array}{l}
                P(AB) = P(A)P(B), \\
                P(BC) = P(B)P(C), \\
                P(AC) = P(A)P(C), \\
                P(ABC) = P(A)P(B)P(C).
            \end{array}
        \right\}
        \Leftrightarrow{}
        A, B, C \text{相互独立}
    \]
\end{define}
\end{document}
