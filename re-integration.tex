\documentclass[main.tex]{subfiles}
\begin{document}
\chapter{重积分}
\[
    \iint _ {D} f(x, y) d\sigma
\]
\section{二重积分的计算法}
\subsection{换元法}
\[
    \iint _ {D} f(x, y) dx dy = \iint _ {D^{'}} f (x(u, v), y(u, v)) |J(u, v)| dudv
\]
其中$J(u, v)$为雅可比式$J(u, v) = \frac{\partial (x, y)}{\partial (u, v)}$
\paragraph{极座标}
\[
    \iint _ {D} f(x, y) dx dy = \iint _ {D^{'}} f (\rho cos\ \theta, \rho sin\ \theta) \rho d\rho d\theta
\]
\begin{example}
    {$\int_0^{2a} dx \int_0^{\sqrt{2ax-x^2}} (x^2 + y^2) dy$}
    \[
        \int_0^{2a} dx \int_0^{\sqrt{2ax-x^2}} (x^2 + y^2) dy
        = \int_0^{\frac{\pi}{2}} d\theta \int_0^{2acos\ \theta} \rho^2 \rho d\rho
    \]
    \begin{center}
        \begin{tikzpicture}[domain=0:4]
            \draw[->] (-0.2,0) -- (2.2,0) node[right] {$x$};
            \draw[->] (0,-0.2) -- (0,1.2) node[above] {$y$};
            \draw[domain=0:2, line width=1pt] plot function{0};
            \draw[domain=0:2, line width=1pt, samples=100] plot function{sqrt(2*x - x**2)};
            \draw[domain=0:2, line width=1pt, dashed] plot function{1};
            \draw (0, 1) node[left] {$a$};
            \draw (2, 0) node[below] {$2a$};
            \draw (1, 0.5) node {$D$};
        \end{tikzpicture}
    \end{center}
\end{example}
\section{三重积分}
\[
    \iiint _ {\Omega} f(x, y, z) dv
\]
\section{重积分的应用}
\subsection{曲面的面积}
对于曲面$z = f(x, y)$面积可写为
\[
    A = \iint _{D} \sqrt{1 + f_x^2 + f_y^2} d\sigma
\]
\subsection{质心}
对于面密度为$\mu (x, y)$的薄片, 质心为
\[
    (\bar{x}, \bar{y}) = (\frac{\iint _{D} x \mu (x, y) d\sigma}{\iint _{D} \mu (x, y) d\sigma}, \frac{\iint _{D} y \mu (x, y) d\sigma}{\iint _{D} \mu (x, y) d\sigma})
\]
若密度均匀
\[
    (\bar{x}, \bar{y}) = (\frac{\iint _{D} x d\sigma}{A}, \frac{\iint _{D} y d\sigma}{A})
\]
$A = \iint _{D} d\sigma$为面积
\end{document}